\documentclass[12pt]{article}
\usepackage{graphicx}

\begin{document}


\begin{flushleft}
\textbf{130}\hspace*{1cm} \texttt{CHAPTER NINE}
\end{flushleft}

\vspace*{0.3cm}
\begin{center}
\section{picture}
\includegraphics[width=14cm,height=8cm]{130.png}

\textbf{FIGURE 9.1} \hspace*{0.1cm} Delphi Entry Form (With permission from Kennedy, 2002.)\\
\end{center}

\vspace*{0.5cm}
There are a number of features of a well-designed Delphi that are illustrated in this example.\\

\vspace*{0.2cm}
\begin{itemize}
  \item The three 5-point Likert-type scales are distributed across the page and the for-matting works well at a variety of text sizes chosen in the respondent's browser.\\
  \item There is a consistent format for all questions used in the study, thus reducing the cognitive load and allowing participants to move accurately and quickly through the survey.\\
  \item Three separate pieces of information are extracted from each question, thus probing more deeply about the issue, without forcing a larger number of ques-tions or screen displays.\\
  \item Each question is downloaded separately from the Web server and the answers can be checked individually for incompleteness (if desired). \\
\end{itemize}

\begin{itemize}
  \item There are a number of navigational aids allowing participants to control the order of the questions (if they so desire) and allowing them to complete the sur-vey at a later, if they are forced to leave the survey before completion.\\
  \item Headings that require ratings are highlighted until they are answered, then the highlighting disappears.\\
  \item A progress meter provides feedback to the respondent on his or her progress through the survey.\\
\end{itemize}

\newpage
\begin{flushright}
 \texttt{NET-BASED CONSENSUS TECHNIQUES} \hspace*{1cm} \textbf{131}
\end{flushright}

\vspace*{0.5cm}
\begin{itemize}
  \item There is a comments box under each question for elaborations or concerns.\\
  \item There is a help button (labeled instructions) on each page of the survey.\\
\end{itemize}

\vspace*{0.3cm}
\large{

\vspace{3mm}
\hspace{-1cm}
\textbf{Distributing Questions}\\
\vspace{3mm}
}

As mentioned previously, the e-researcher may distribute background materials to par-ticipants, but this is generally not necessary. Usually the participants have been chosen because of their expertise and thus are likely not in need of background briefing. Fur-ther, the selection of background information may result in undue influence by the researcher in the assessment of the participants' views and opinions. In addition, ready ing such material will add to the amount of time required to participate in the study and often results in only some of the members actually reading the material. The advantages and disadvantages of including the questions in an email survey as opposed to inviting participants to complete the questions online are the same as those dis-cussed in the survey chapter (see Chapterll). Either method can work effectively, and the decision may rest on the skills and access of the researcher to a sophisticated Web development and delivery environment.\\

\vspace*{0.3cm}

\large{
\vspace{3mm}
\hspace{-1cm}
\textbf{Aggregating and Returning Results}\\
\vspace{3mm}
}

Since the goal of consensus research is either to revel the degree of consensus or to move the group toward consensus, feedback is provided between each round of ques-tions, documenting and sometimes quantifying group progress towards consensus. There are a number of techniques for illustrating the central value and the amount of spread around the degree of consensus. These include the mean and standard devia-tion, as well as the median and interquartile range. The median and IQR statistics are less affected by extreme answers or outliers and less sensitive to skew, and thus are most often used with Delphi Methods. Even more sophisticated statistical techniques can be used including multivariate techniques, such as Multi-Dimensional Scaling that illus-trate relationship and similarity of opinion in multidimensional apaces (T uroff et al.,1995).\\


\hspace{0.5cm} Figure 9.2 provides an example of a portion of the information returned to par-ticipants after the second round of questioning. This information appeared on the screen directly below Figure 1 and can be scrolled to or linked directly from the but-tons at the bottom of Figure 9.1. Note that a separate graph is provided for each of the three Likert-type scales associated with each question.\\
\hspace{0.5cm} Features of an effective Delphi feedback from include:\\

\vspace*{0.3cm}
\begin{itemize}
  \item The question is repeated to refresh participants' memory of the exact wording of the question to which group answers are illustrated.\\
  \item A histogram is provided showing the results of the three different expert sub-panels who are used as participant groups in this study (academics, administra-tors, and information technology professionals).\\
  \item The mean for each group is calculated and graphically displayed.\\
  \item There is a legend (top right) to explain headings.\\
\end{itemize}

\newpage
\begin{flushleft}
\textbf{132}\hspace*{1cm} \texttt{CHAPTER NINE}
\end{flushleft}

\vspace*{0.3cm}
\begin{center}
\section{picture}
\includegraphics[width=14cm,height=8cm]{132.png}

\textbf{FIGURE 9.2} \hspace*{0.1cm} Statistical Analysis Response \\
\end{center}

\vspace*{0.3cm}
\begin{itemize}
  \item The total number of responses is provided along with scores. Note the list-ing of standard deviation (SD) in the left comlumn.\\
  \item In this example, Kennedy (2002) has not chosen to provide feedback to the par-ticipants that identifies their answers from the previous round, but rather pro-vides the distribution of all responses by the participant's group and the other two subpanel groups. A copy of the participant's previous responses is provided by email, if requested.\\
\end{itemize}

\vspace*{0.2cm}
Besides providing descriptive information to the participants indicating their own and the group's responses, these additional rounds can be used to modify or clarify the original questions, using feedback from the earlier round participants.\\
\hspace{0.5cm} Finally, we note that words (as well as numeric displays of divergence) are also critically important in reaching consensus. In this example, Kennedy chose to extract and list these comments and arguments that are accessible in subsequent rounds by activating the comments button in Figure 9.1. She removed all personal information that could identify the respondents except the subpanel group to which they belong.Outliers are also identified with an asterisk.\\

\end{document} 